\begin{eabstract}

Managers of a large number of physical computers, such as a large data center, are faced with a situation in which computers are analyzed in addition to normal surveillance and investigation of causes in an emergency.
However, if you rent a computer to a customer, you often do not have root privileges.

Existing monitoring methods include monitoring software that starts as a process on the monitored host and a method that starts and debugs as a VM.
In an emergency, for example, when a kernel panic occurs, the core dump is analyzed.

However, in managing a large number of physical computers, it is difficult to introduce various solutions to address the problem.

Therefore, in this research, under the NetTLP environment, which is a debugging environment that operates even if the operating system is stopped as long as the power is on, for a large number of computers,
The purpose of this study is to perform a memory search using the implementation of RDMA implemented in NetTLP and to restore the context of the operating system of a physical computer over a running network.

As an implementation and evaluation, we show that, with limited information, a process list of the monitored host of 64bit Linux over the network can be obtained from the host, restored, and output.

When restoring a process list by memory search using RDMA, in this research, search is performed starting from init\_task, which is a process having a process ID of 0.
init\_task holds a task\_struct structure, but the type information of the task\_struct structure running on the monitored host must be inferred from the kernel config.
In this research, acquisition of type information of task\_struct structure, that is, calculation of offset of each field,
This is achieved by analyzing the non-atomic memory dump on the host executing the implementation, collecting the kernel configuration values, and rebuilding.

\end{eabstract}
