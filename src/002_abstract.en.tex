\begin{eabstract}

Managers of a large number of physical computers, such as a large data center, are faced with a situation in which computers are analyzed in addition to normal surveillance and investigation of causes in an emergency.
However, if you rent a computer to a customer, you often do not have root privileges.

Existing monitoring methods include monitoring software that starts as a process on the monitored host and a method that starts and debugs as a VM.
In an emergency, for example, when a kernel panic occurs, the core dump is analyzed.

However, in managing a large number of physical computers, it is difficult to introduce various solutions to address the problem.

Therefore, in this research, under the NetTLP \ cite {246316} environment, which is a debugging environment that operates even if the operating system is stopped as long as the power is on, for a large number of computers,
The purpose of this study is to perform a memory search using the RDMA function implemented in NetTLP, and to restore the context of the operating system of a physical computer over a running network.

As an implementation and evaluation for the purpose, we show that the process list of the monitored host can be restored with limited information.

Also, when restoring the process list from only operating system version information and non-atomic memory dumps,
A method to derive internal values ​​that differ for each machine and to absorb the differences for each machine was also shown.

\end{eabstract}
