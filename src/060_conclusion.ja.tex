\chapter{まとめと結論}
\label{chap:conclusion}

\section{まとめ}

本論文のまとめとして,各章の内容を述べる.

\ref{chap:introduction}章では、本研究の背景および課題として,大規模データセンタのコンピュータ管理者にとって,
様々な設定をもつ大量の物理的なコンピュータの監視および解析が困難であることを述べた
そこで本研究の目的として,

\ref{chap:related_works}章では、仮想環境によるオペレーティングシステムのデバッグや,監視対象ホスト内で監視プロセスを起動する手法など,
既存のオペレーティングシステムのコンテキストを監視する手法について述べた.
さらに,様々なRDMA実装がある中で,本研究で使用するNetTLPによるRDMAを使う理由について述べた.

\ref{chap:approach}章では、動作中のコンピュータから取得したアトミックではないメモリダンプから
オペレーティングシステムのコンテキストを復元するための手法について述べた.
その上で,メモリからどのような情報を探索することで,コンピュータの状態を復元できるのかについて述べた.
さらに,物理メモリアドレスのみを指定できる中で,取得が困難な情報がどのような種類の情報で,その情報を本研究においてどのように復元していくかについて述べた.

\ref{chap:implementation}章では、\ref{chap:approach}章で述べた手法を実現するための具体的な実装について述べた.
特にメモリダンプしかない状態からいかにして,監視対象ホストのカーネルコンフィグの値を復元するか,復元した値からコンピュータの内部的な値,
すなわち構造体のオフセットを復元する実装について述べた.
最終的に,Linuxカーネルのバージョンのみを通知された状態から,プロセス一覧に関する情報を取得するために必要な値を復元し,プロセス一覧を出力できる実装について述べた.

\ref{chap:evaluation}章では、本研究における評価として,Linuxカーネルのバージョンのみが与えられた状態でプロセスリストの一覧を取得できることを示した.
実験として,\ref{chap:implementation}章で述べた工程を一つずつ実行した過程を示した.
最終的に,本研究の実装の出力結果と監視対象ホストで実行したpsコマンドの出力結果を比較し,任意に起動したプロセスのIDが等しくなっていることを示した.

\section{結論}

本研究の結論として,\ref{chap:implementation}章で述べた実装を用いることで,

結論は俺の知見,考察.結局こうだったを述べる.こういうところでは使えた,こういうところではダメだった.

本論文の結論として,\ref{chap:implementation}で述べたような実装を用いることで,目的を達成することができた.(加筆)

\section{今後の課題}

特にカーネルバージョンの特定.

\subsection{セキュリティ的な課題}

課題
