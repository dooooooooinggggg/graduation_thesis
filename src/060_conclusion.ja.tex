\chapter{まとめと結論}
\label{chap:conclusion}

\section{まとめ}

各章の振り返り

\section{結論}

結論は俺の知見,考察.結局こうだったを述べる.こういうところでは使えた,こういうところではダメだった.

\section{今後の課題}

特にカーネルバージョンの特定とKASLRのバイパスについて書く.

\subsection{KASLR}

これは本来,ASLRに実装されたものであり,それのカーネルバージョンである.
内部からの攻撃に向けた防御策であるため,この研究の手法であればバイパスは可能であると考えられる.
なぜなら,メモリ保護機能,すなわちアクセスできるメモリアドレスの制限を受けないレイヤーで動作する環境だから.
