\chapter{まとめと結論}
\label{chap:conclusion}

\section{まとめ}

各章の振り返り

本論文の各章を振り返る

\ref{chap:related_works}章では、本研究で使用するRDMAを使う理由について述べた.

\ref{chap:approach}章では、メモリの情報からオペレーティングシステムのコンテキストを復元するということについて述べた.
その上で,メモリからどのような情報を探すことで,オペレーティングシステム,コンピュータの状態を復元することができるのかという点について述べた.

\ref{chap:implementation}章では、本研究で実装したものについて述べた(加筆)

\ref{chap:evaluation}章では、本研究における評価として,Linuxカーネルのバージョンのみが与えられた状態でプロセスリストの一覧を取得できることを示した.

\section{結論}

結論は俺の知見,考察.結局こうだったを述べる.こういうところでは使えた,こういうところではダメだった.

本論文の結論として,\ref{chap:implementation}で述べたような実装を用いることで,目的を達成することができた.(加筆)

\section{今後の課題}

特にカーネルバージョンの特定とKASLRのバイパスについて書く.

\subsection{KASLR}

KASLRは本来,ASLRとして実装されたものであり,それのカーネルバージョンである.
内部からの攻撃に向けた防御策であるため,この研究の手法であればバイパスは可能であると考えられる.
なぜなら,メモリ保護機能,すなわちアクセスできるメモリアドレスの制限を受けないレイヤーで動作する環境だから.

\subsection{セキュリティ的な課題}

課題
