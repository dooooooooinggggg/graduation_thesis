\chapter{実装}
\label{chap:implementation}

実装がまだ完成してないので,空想です.

\section{実装の全体}

ダンプしてくるということを書く.物理アドレスのマッピングに関しても

次の工程として,収集したカーネルコンフィグを元に手元のコンピュータでLinuxカーネルのソースコードに対してプリプロセスの処理を行い,task_struct型を確定する.
さらに,ソースコード上にある__phys_addrの実体を収集する.

最後に,この工程で得られた情報をもとに,libtlpで提唱されている手法を用いて,プロセスの一覧を正しく取得できることを確認する.

\section{工程1}

第一の工程として,RDMA NICを用いて,監視対象ホストのメモリを全探索し,メモリに落ちているSystem.mapのうち,init_taskが配置されている仮想アドレス空間に関する情報と,
Linuxカーネルにおける__phys_addr関数,task_struct型を決定するためのカーネルコンフィグに関する情報を収集することは\ref{chap:related_works}章で述べた.

(本セクションでは,その実装を詳しく書く.ダンプしまくるやつの説明をここに書く)

\section{工程2}

収集したカーネルコンフィグを元に手元のコンピュータでLinuxカーネルのソースコードに対してプリプロセスの処理を行い,task_struct型を確定する.
さらに,ソースコード上にある__phys_addrの実体を収集する部分に関する実装をより詳しく書く.

\section{工程3}

最後に,集められたデータをもとに,process-listを改造したものに関する説明をここに書く
