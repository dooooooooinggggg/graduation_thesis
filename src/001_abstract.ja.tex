\begin{jabstract}

大規模データセンターなど,大量の物理的なコンピュータの管理者は,通常時の監視に加えて緊急時の原因究明など,コンピュータを解析する場面に直面する.
しかし,顧客にコンピュータを貸し出している場合は,root権限がないことも多い.

既存の監視手法では,監視対象ホストのプロセスとして起動する監視ソフトウェアや,VMとして起動しデバッグを行う手法がある.
また,緊急時,例えばカーネルパニックがおきた際はコアダンプの解析を行う.

しかし,大量の物理的なコンピュータを管理するにあたり,問題に対処するための様々な解決策を導入することは難しい.

そこで本研究では,大量のコンピュータに対して,電源さえ入っていればオペレーティングシステムが停止していても動作するデバッグ環境であるNetTLPがプログラムされたFPGAボードを監視対象ホストに物理的に設置するだけで,
NetTLPに実装されたRDMAの実装を用いてメモリ探索を行い,動作中のネットワーク越しにある物理的なコンピュータのオペレーティングシステムのコンテキストを復元することを目的とする.

実装および評価として,限られた情報の中で,ネットワーク越しにある64bit Linuxの監視対象ホストのプロセス一覧を,自ホストから取得し,復元,出力できることを示す.

RDMAを用いたメモリ探索でプロセス一覧を復元するに際し,本研究ではプロセスIDとして0を持つプロセスであるinit\_taskを起点として探索を行う.
init\_taskはtask\_struct構造体を保持しているが,監視対象ホストで動作中のtask\_struct構造体の型情報はカーネルコンフィグから推測するほかない.
本研究では,task\_struct構造体の型情報の取得,すなわち各フィールドのオフセットの算出を,
実装したプログラムを実行するホストにてアトミックではないメモリダンプを解析しカーネルコンフィグの値を収集,再度ビルドすることで達成している.

\end{jabstract}
