\begin{jabstract}

大規模データセンターなど,大量の物理的なコンピュータの管理者は,通常時の監視に加えて緊急時の原因究明など,コンピュータを解析する場面に直面する.
しかし,顧客にコンピュータを貸し出している場合は,root権限がないことも多い.

既存の監視手法では,監視対象ホストのプロセスとして起動する監視ソフトウェアや,VMとして起動しデバッグを行う手法がある.
また,緊急時,例えばカーネルパニックがおきた際はコアダンプの解析を行う.
しかし,大量のコンピュータを管理するにあたり,それぞれのコンピュータに上記の対策を施すことは難しい.
なぜなら,コンピュータによってオペレーティングシステムの設定が異なることがあるからである.

そこで本研究では,大量のコンピュータに対して,電源さえ入っていればオペレーティングシステムが停止していても動作するデバッグ環境であるNetTLP\cite{246316}上で,
各コンピュータにおけるオペレーティングシステムのビルドコンフィグの差異を吸収する実装と評価を行った.

\end{jabstract}
