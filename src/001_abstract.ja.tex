\begin{jabstract}

大規模データセンターなど,大量の物理的なコンピュータの管理者は,通常時の監視に加えて緊急時の原因究明など,コンピュータを解析する場面に直面する.
しかし,顧客にコンピュータを貸し出している場合は,root権限がないことも多い.

既存の監視手法では,監視対象ホストのプロセスとして起動する監視ソフトウェアや,VMとして起動しデバッグを行う手法がある.
また,緊急時,例えばカーネルパニックがおきた際はコアダンプの解析を行う.

しかし,大量の物理的なコンピュータを管理するにあたり,問題に対処するための様々な解決策を導入することは難しい.

そこで本研究では,大量のコンピュータに対して,電源さえ入っていればオペレーティングシステムが停止していても動作するデバッグ環境であるNetTLP\cite{246316}環境の元,
NetTLPに実装されたRDMAの機能を用いてメモリ探索を行い,動作中のネットワーク越しにある物理的なコンピュータのオペレーティングシステムのコンテキストを復元することを目的とする.

目的に対する実装および評価として,限られた情報の中で,監視対象ホストのプロセス一覧を復元できることを示す.

また,オペレーティングシステムのバージョン情報およびアトミックではないメモリダンプのみからプロセス一覧を復元するに際して,
マシンごとに異なる内部的な値を導出し,マシンごとの差異を吸収するための手法も示した.

\end{jabstract}
