\chapter{評価}
\label{chap:evaluation}

本章では,本研究における実装によって,正しく監視対象ホストの状態を取得できているかどうかを評価とする.
また,hoge,fugaな時にもその評価が正しくできているかを確認する.

\section{評価手法}

評価手法として,カーネルのバージョンのみわかる状態から,正しくps auxと同じような出力を得られるかどうか,
実験用に起動したプロセスを,本研究の実装上から確認できるかどうかを評価とする.
また,実験の最中に導出した値が実際のホストにおいて正しいかどうかを確認し,それを評価とする.

プロセスとして監視を行う手法と,本研究の実装を,通常稼働中とカーネルパニック発生時における実行の可否について述べる.

\section{実験環境}

本研究では,以下の環境で実験を行う.

\begin{table}[htbp]
    % \label{tab:}
    \caption{実装を実行するホスト}
    \begin{center}
      \begin{tabular}{ll}
      \hline
      \verb|Linuxカーネルのバージョン| Linux 4.19.0-6-amd64 \\
      \hline
      \verb|ディストリビューション| Debian buster 10.2 \\
      \hline
      \end{tabular}
    \end{center}
  \end{table}

  \begin{table}[htbp]
    % \label{tab:}
    \caption{監視対象ホスト}
    \begin{center}
      \begin{tabular}{ll}
      \hline
      \verb|Linuxカーネルのバージョン| Linux 4.15.0-74-generic \\
      \hline
      \verb|ディストリビューション| Ubuntu 18.04.3 LTS (Bionic Beaver) \\
      \hline
      \end{tabular}
    \end{center}
  \end{table}

\section{評価手順}
\label{section:eval}

評価手順として,\ref{chap:implementation}章で述べた実装を用いて,実際に全ての工程を,手順に沿って実行していく.

\subsection{前提}

前述したように,本研究の実験においては,実装を実行するホストは,監視対象ホストに関して,Linuxカーネルのバージョンのみを情報として保持する.

\subsection{メモリダンプの取得}
\label{subsection:eval_dump_mem}

\ref{section:mem_dump}で述べた実装である,dump_memを用いて,メモリダンプを取得する.

\begin{itembox}[l]{実行方法}
    \begin{verbatim}
./dump_mem > dump
    \end{verbatim}
\end{itembox}

このファイルを実行すると,搭載している物理メモリの大きさに等しい,8GBのファイルが作成される.

\begin{itembox}[l]{./dump_mem}
    \begin{verbatim}
$ ./dump_mem > dump
$ ls -alh
# total 8.0G
# drwxr-xr-x  2 tatsunori tatsunori 4.0K Jan 26 23:01 .
# drwxr-xr-x 17 tatsunori tatsunori 4.0K Jan 26 19:42 ..
# -rw-r--r--  1 tatsunori tatsunori 8.0G Jan 26 19:54 dump
    \end{verbatim}
\end{itembox}

このファイルを以後,メモリダンプと呼ぶ.

\subsection{カーネルコンフィグを復元する}
\label{subsection:eval_restore_kconfig}

取得してきたメモリダンプに対して,\ref{section:restore_kconfig}で述べたように,処理を施し,ビルド時のカーネルコンフィグを復元する.

\begin{itembox}[l]{strings}
    \begin{verbatim}
strings dump | grep CONFIG > str_list
    \end{verbatim}
\end{itembox}

\begin{itembox}[l]{exec restore_kconfig.py}
    \begin{verbatim}
python restore_kconfig.py str_list > restored_kconfig
    \end{verbatim}
\end{itembox}

ここで生成されたresotred_kconfigを,実装を実行するホストでカーネルをビルドする際に,.configとしてそのまま用いる.

\subsection{復元したカーネルをプリプロセッサに通す}

\ref{section:preprocess}で述べたように,\ref{subsection:eval_restore_kconfig}の結果得られたカーネルコンフィグを用いて,
カーネルのビルドを実装を実行するホストで行う.

\begin{itembox}[l]{build Linux kernel}
    \begin{verbatim}
cd /path/to/linux-source-4.15.0
cp /path/to/restored_kconfig .config
    \end{verbatim}
\end{itembox}

その際に,\ref{section:preprocess}で述べたように,プリプロセッサによる処理である中間ファイルを残す設定とするため,
Makefileに変更を加える.本研究では,監視対象ホストのバージョンは,Linux 4.15.0-74-genericでありその変更は,\ref{section:preprocess}で述べたものと同じである.

変更したのちに,以下のコマンドを実行し,ビルドを開始する.

\begin{itembox}[l]{ビルド}
    \begin{verbatim}
make -j10
    \end{verbatim}
\end{itembox}

ビルドが終了すると,ソースコードが中間ファイルの生成によって以下のようなサイズとなる.

\begin{itembox}[l]{ビルド}
    \begin{verbatim}
$ du -shc linux-source-4.15.0
64G	linux-source-4.15.0
64G	total
    \end{verbatim}
\end{itembox}

\subsection{実行環境におけるinit_taskの先頭アドレス}
\label{subsection:eval_init_task_head}

\ref{section:define_task_struct}で述べた内容に基づいて,作成したprint_offsetを実行した結果は以下となった.

\begin{itembox}[l]{名前考える}
    \begin{verbatim}
$ ./print_offset_restore
# task_struct size: 9088

# state: 16
# pid: 2216
# children: 2248
# sibling: 2264
# comm: 2640
# real_parent: 2232
    \end{verbatim}
\end{itembox}

この結果をもとに,\ref{subsection:eval_init_task_head}にて,init_taskの先頭アドレスを算出する.

\ref{subsection:eval_dump_mem}で取得したメモリダンプから,\ref{subsection:find_init_task}で述べたように,swapper/0という文字列を以下のコマンドで検索を行う.

\begin{itembox}[l]{find swapper/0}
    \begin{verbatim}
$ xxd dump | grep swapper/0
# 023f3ed0: 7377 6170 7065 722f 3000 0000 0000 0000  swapper/0.......
# e09f3ed0: 7377 6170 7065 722f 3000 0000 0000 0000  swapper/0.......
    \end{verbatim}
\end{itembox}

以上の結果となった.このうち一つ目の値を取り出すと,023f3ed0であるが,\ref{subsection:find_init_task}に倣って,
0x023f3ed0の10進表記である37699280から,commフィールドのオフセットである2640を減算し,128KBのバイト数である131072を加算する.
その結果,得られた値は,37699280-2640+131072=37827712 となる.
この値は物理アドレスであるため,これをカーネルの仮想アドレスに変換する.

\begin{itembox}[l]{__phys_addr_nodebug()}
    \begin{verbatim}
#define __START_KERNEL_map _AC(0xffffffff80000000, UL)
#define phys_base 0x0 /* x86 */

static inline unsigned long __phys_addr_nodebug(unsigned long x)
{
    // unsigned long before = x;
    unsigned long y = x - __START_KERNEL_map;

    /* use the carry flag to determine if x was < __START_KERNEL_map */
    x = y + ((x > y) ? phys_base : (__START_KERNEL_map - PAGE_OFFSET));

    // printf("0x%lx -> 0x%lx(%luGB)\n", before, x, gb_from_lx(x));

    return x;
}
    \end{verbatim}
\end{itembox}

Linuxカーネルで物理アドレスが0からのストレートマップとなるのは,上述のソースコード内における三項演算子のうち,条件式が真となる場合である.
この関数から,37827712の16進数表記である0x2413480という結果が帰る場合は条件式が真となる場合であるため,カーネル空間の仮想アドレスにおいて,
37827712という物理アドレスに対応する仮想アドレスは, 0x2413480 + phys\_base + 0xffffffff80000000 の結果である 0xffffffff82413480 となり,
これが本研究における監視対象ホストのinit_taskの仮想アドレスと推定する.
また,この値をprocess-list.cの引数として使用する.

\subsection{process-list.cの実行}

\ref{subsection:eval_init_task_head}で導いた値を引数として,以下のようにコマンドを実行する.
実行結果については,\ref{subsection:exec_process_list}で述べる.

\begin{itembox}[l]{process-list.cの実行}
    \begin{verbatim}
./process-list 0xFFFFFFFF82413480
    \end{verbatim}
\end{itembox}

\section{評価}

\subsection{値が正しいこと}

\ref{section:eval}における評価手順において,導出した値として,init_taskの仮想アドレスが正しいかどうかを評価する.
評価手法としては,実験の際に導いた0xffffffff82413480という値が監視対象ホストの値と等しいかどうかを,監視対象ホストのkallsymsを参照することで比較する.

比較結果は以下であり,導出した0xffffffff82413480という値が正しい値であることを示した.

\begin{itembox}[l]{kallsymsの出力}
    \begin{verbatim}
$ sudo cat kallsyms | grep "D init_task"
# ffffffff82413480 D init_task
    \end{verbatim}
\end{itembox}

\subsection{通常稼働中における評価}
\label{subsection:exec_process_list}

\ref{section:eval}によって求めた値を用いて,process-list.cを実行する.
実行結果は以下である.

\begin{itembox}[l]{process-listの出力}
    \begin{verbatim}
$ ./process-list 0xFFFFFFFF82413480

init_vm_addr: 0xffffffff82413480
PhyAddr             PID STAT COMMAND
0x00000002413480      0 R: swapper/0
0x000002361f0000      1 S: systemd
0x0000022ea616c0    287 S: systemd-journal
0x0000022da0ad80    297 S: blkmapd
0x0000022e232d80    308 S: systemd-udevd
0x000002333c8000    527 S: systemd-timesyn
0x000002333cc440    530 S: rpcbind
0x00000233d72d80    534 S: cron
0x00000233d70000    536 S: atd
0x0000022e732d80    545 S: rsyslogd
0x0000022dbfad80    556 S: irqbalance
0x0000022dbf96c0    561 S: accounts-daemon
0x0000022dbfdb00    569 S: dbus-daemon
0x0000022f752d80    586 S: wpa_supplicant
0x0000022f754440    589 S: systemd-logind
0x0000022ea40000    592 S: networkd-dispat
0x0000022f7e0000    607 S: polkitd
0x0000022f750000    634 S: systemd-resolve
0x0000022da08000    663 S: dhclient
0x00000235378000    777 S: nmbd
0x00000234bac440    781 S: unattended-upgr
0x00000234baad80    783 S: sshd
0x00000234ba2d80   3009 S: sshd
0x0000022e735b00   3142 S: sshd
0x00000234ba16c0   3143 S: zsh
0x00000234ba8000    787 S: agetty
0x00000234ba0000    819 S: smbd
0x00000234452d80    821 S: smbd-notifyd
0x000002344516c0    822 S: cleanupd
0x00000234450000    823 S: lpqd
0x0000022f7e4440   3032 S: systemd
0x0000022f7516c0   3033 S: (sd-pam)
0x000002361f5b00      2 S: kthreadd
0x000002361f16c0      4 D: kworker/0:0H
0x0000023622ad80      6 D: mm_percpu_wq
0x000002362296c0      7 S: ksoftirqd/0
0x0000023622c440      8 D: rcu_sched
0x00000236228000      9 D: rcu_bh
0x0000023622db00     10 S: migration/0
0x00000236254440     11 S: watchdog/0
0x000002362516c0     12 S: cpuhp/0
0x0000023625c440     13 S: cpuhp/1
0x00000236258000     14 S: watchdog/1
0x0000023625db00     15 S: migration/1
0x0000023625ad80     16 S: ksoftirqd/1
0x0000023630db00     18 D: kworker/1:0H
0x0000023630ad80     19 S: cpuhp/2
0x000002363096c0     20 S: watchdog/2
0x0000023630c440     21 S: migration/2
0x00000236308000     22 S: ksoftirqd/2
0x00000236372d80     24 D: kworker/2:0H
0x000002363716c0     25 S: cpuhp/3
0x00000236374440     26 S: watchdog/3
0x00000236370000     27 S: migration/3
0x000002363dad80     28 S: ksoftirqd/3
0x000002363dc440     30 D: kworker/3:0H
0x00000235c54440     31 S: kdevtmpfs
0x00000235c9ad80     32 D: netns
0x00000235c996c0     33 S: rcu_tasks_kthre
0x00000235c9c440     34 S: kauditd
0x00000235c98000     35 D: kworker/0:1
0x00000235d596c0     37 S: khungtaskd
0x00000235d5c440     38 S: oom_reaper
0x00000235d6ad80     39 D: writeback
0x00000235d696c0     40 S: kcompactd0
0x00000235d6c440     41 S: ksmd
0x00000235d68000     42 S: khugepaged
0x000002363d8000     43 D: crypto
0x000002363ddb00     44 D: kintegrityd
0x00000235d7db00     45 D: kblockd
0x00000235d7ad80     46 D: kworker/2:1
0x00000235d796c0     47 D: kworker/3:1
0x00000235d7c440     48 D: ata_sff
0x00000235d78000     49 D: md
0x00000235eb0000     50 D: edac-poller
0x00000235eb5b00     51 D: devfreq_wq
0x00000235eb2d80     52 D: watchdogd
0x00000235c52d80     55 S: kswapd0
0x00000235c516c0     56 D: kworker/u9:0
0x0000022ea05b00     57 S: ecryptfs-kthrea
0x00000235eb16c0     99 D: kthrotld
0x00000235eb4440    100 D: acpi_thermal_pm
0x0000022ea3db00    105 D: kworker/1:2
0x0000022ea72d80    109 D: ipv6_addrconf
0x0000022ea696c0    118 D: kstrp
0x0000022ea6db00    135 D: charger_manager
0x0000022ea016c0    181 S: scsi_eh_0
0x0000022ea04440    182 D: scsi_tmf_0
0x0000022ea02d80    183 S: scsi_eh_1
0x0000022ea38000    184 D: scsi_tmf_1
0x0000022e2316c0    185 S: scsi_eh_2
0x0000022e234440    186 D: scsi_tmf_2
0x0000022e230000    187 S: scsi_eh_3
0x0000022ea45b00    188 D: scsi_tmf_3
0x0000022ea316c0    189 S: scsi_eh_4
0x0000022ea3ad80    190 D: scsi_tmf_4
0x0000022ea396c0    191 S: scsi_eh_5
0x0000022ea3c440    192 D: scsi_tmf_5
0x0000022ea34440    198 D: e1000e
0x0000022ea516c0    199 D: e1000e
0x0000022ea52d80    201 D: kworker/1:1H
0x0000022da0db00    230 D: kworker/3:1H
0x0000022ea6ad80    232 S: jbd2/sda1-8
0x00000235d5ad80    233 D: ext4-rsv-conver
0x00000235d58000    243 D: kworker/0:1H
0x0000022ea716c0    260 D: kworker/2:1H
0x0000022e734440    266 D: rpciod
0x0000022e7316c0    267 D: xprtiod
0x0000022ea6c440    366 D: ttm_swap
0x0000022f91db00   1920 D: kworker/1:0
0x00000233d75b00   2843 D: kworker/0:0
0x0000022ea65b00   2844 D: kworker/3:0
0x0000022f7e5b00   3520 D: kworker/u8:1
0x00000234455b00   3555 D: kworker/2:0
0x00000234ba5b00   3592 D: kworker/u8:2
    \end{verbatim}
\end{itembox}

\begin{itembox}[l]{監視対象ホストにおけるpsコマンドの結果}
    \begin{verbatim}
$ ps aux
USER       PID %CPU %MEM    VSZ   RSS TTY      STAT START   TIME COMMAND
root         1  0.0  0.1 225484  9196 ?        Ss   Jan27   0:14 /sbin/init nopti nospectre_v2 nokaslr
root         2  0.0  0.0      0     0 ?        S    Jan27   0:00 [kthreadd]
root         4  0.0  0.0      0     0 ?        I<   Jan27   0:00 [kworker/0:0H]
root         6  0.0  0.0      0     0 ?        I<   Jan27   0:00 [mm_percpu_wq]
root         7  0.0  0.0      0     0 ?        S    Jan27   0:00 [ksoftirqd/0]
root         8  0.0  0.0      0     0 ?        I    Jan27   0:02 [rcu_sched]
root         9  0.0  0.0      0     0 ?        I    Jan27   0:00 [rcu_bh]
root        10  0.0  0.0      0     0 ?        S    Jan27   0:00 [migration/0]
root        11  0.0  0.0      0     0 ?        S    Jan27   0:00 [watchdog/0]
root        12  0.0  0.0      0     0 ?        S    Jan27   0:00 [cpuhp/0]
root        13  0.0  0.0      0     0 ?        S    Jan27   0:00 [cpuhp/1]
root        14  0.0  0.0      0     0 ?        S    Jan27   0:00 [watchdog/1]
root        15  0.0  0.0      0     0 ?        S    Jan27   0:00 [migration/1]
root        16  0.0  0.0      0     0 ?        S    Jan27   0:00 [ksoftirqd/1]
root        18  0.0  0.0      0     0 ?        I<   Jan27   0:00 [kworker/1:0H]
root        19  0.0  0.0      0     0 ?        S    Jan27   0:00 [cpuhp/2]
root        20  0.0  0.0      0     0 ?        S    Jan27   0:00 [watchdog/2]
root        21  0.0  0.0      0     0 ?        S    Jan27   0:00 [migration/2]
root        22  0.0  0.0      0     0 ?        S    Jan27   0:00 [ksoftirqd/2]
root        24  0.0  0.0      0     0 ?        I<   Jan27   0:00 [kworker/2:0H]
root        25  0.0  0.0      0     0 ?        S    Jan27   0:00 [cpuhp/3]
root        26  0.0  0.0      0     0 ?        S    Jan27   0:00 [watchdog/3]
root        27  0.0  0.0      0     0 ?        S    Jan27   0:00 [migration/3]
root        28  0.0  0.0      0     0 ?        S    Jan27   0:00 [ksoftirqd/3]
root        30  0.0  0.0      0     0 ?        I<   Jan27   0:00 [kworker/3:0H]
root        31  0.0  0.0      0     0 ?        S    Jan27   0:00 [kdevtmpfs]
root        32  0.0  0.0      0     0 ?        I<   Jan27   0:00 [netns]
root        33  0.0  0.0      0     0 ?        S    Jan27   0:00 [rcu_tasks_kthre]
root        34  0.0  0.0      0     0 ?        S    Jan27   0:00 [kauditd]
root        35  0.0  0.0      0     0 ?        I    Jan27   0:00 [kworker/0:1]
root        37  0.0  0.0      0     0 ?        S    Jan27   0:00 [khungtaskd]
root        38  0.0  0.0      0     0 ?        S    Jan27   0:00 [oom_reaper]
root        39  0.0  0.0      0     0 ?        I<   Jan27   0:00 [writeback]
root        40  0.0  0.0      0     0 ?        S    Jan27   0:00 [kcompactd0]
root        41  0.0  0.0      0     0 ?        SN   Jan27   0:00 [ksmd]
root        42  0.0  0.0      0     0 ?        SN   Jan27   0:00 [khugepaged]
root        43  0.0  0.0      0     0 ?        I<   Jan27   0:00 [crypto]
root        44  0.0  0.0      0     0 ?        I<   Jan27   0:00 [kintegrityd]
root        45  0.0  0.0      0     0 ?        I<   Jan27   0:00 [kblockd]
root        46  0.3  0.0      0     0 ?        I    Jan27   2:13 [kworker/2:1]
root        47  0.0  0.0      0     0 ?        I    Jan27   0:00 [kworker/3:1]
root        48  0.0  0.0      0     0 ?        I<   Jan27   0:00 [ata_sff]
root        49  0.0  0.0      0     0 ?        I<   Jan27   0:00 [md]
root        50  0.0  0.0      0     0 ?        I<   Jan27   0:00 [edac-poller]
root        51  0.0  0.0      0     0 ?        I<   Jan27   0:00 [devfreq_wq]
root        52  0.0  0.0      0     0 ?        I<   Jan27   0:00 [watchdogd]
root        55  0.0  0.0      0     0 ?        S    Jan27   0:00 [kswapd0]
root        56  0.0  0.0      0     0 ?        I<   Jan27   0:00 [kworker/u9:0]
root        57  0.0  0.0      0     0 ?        S    Jan27   0:00 [ecryptfs-kthrea]
root        99  0.0  0.0      0     0 ?        I<   Jan27   0:00 [kthrotld]
root       100  0.0  0.0      0     0 ?        I<   Jan27   0:00 [acpi_thermal_pm]
root       105  0.0  0.0      0     0 ?        I    Jan27   0:00 [kworker/1:2]
root       109  0.0  0.0      0     0 ?        I<   Jan27   0:00 [ipv6_addrconf]
root       118  0.0  0.0      0     0 ?        I<   Jan27   0:00 [kstrp]
root       135  0.0  0.0      0     0 ?        I<   Jan27   0:00 [charger_manager]
root       181  0.0  0.0      0     0 ?        S    Jan27   0:00 [scsi_eh_0]
root       182  0.0  0.0      0     0 ?        I<   Jan27   0:00 [scsi_tmf_0]
root       183  0.0  0.0      0     0 ?        S    Jan27   0:00 [scsi_eh_1]
root       184  0.0  0.0      0     0 ?        I<   Jan27   0:00 [scsi_tmf_1]
root       185  0.0  0.0      0     0 ?        S    Jan27   0:00 [scsi_eh_2]
root       186  0.0  0.0      0     0 ?        I<   Jan27   0:00 [scsi_tmf_2]
root       187  0.0  0.0      0     0 ?        S    Jan27   0:00 [scsi_eh_3]
root       188  0.0  0.0      0     0 ?        I<   Jan27   0:00 [scsi_tmf_3]
root       189  0.0  0.0      0     0 ?        S    Jan27   0:01 [scsi_eh_4]
root       190  0.0  0.0      0     0 ?        I<   Jan27   0:00 [scsi_tmf_4]
root       191  0.0  0.0      0     0 ?        S    Jan27   0:00 [scsi_eh_5]
root       192  0.0  0.0      0     0 ?        I<   Jan27   0:00 [scsi_tmf_5]
root       198  0.0  0.0      0     0 ?        I<   Jan27   0:00 [e1000e]
root       199  0.0  0.0      0     0 ?        I<   Jan27   0:00 [e1000e]
root       201  0.0  0.0      0     0 ?        I<   Jan27   0:00 [kworker/1:1H]
root       230  0.0  0.0      0     0 ?        I<   Jan27   0:00 [kworker/3:1H]
root       232  0.0  0.0      0     0 ?        S    Jan27   0:00 [jbd2/sda1-8]
root       233  0.0  0.0      0     0 ?        I<   Jan27   0:00 [ext4-rsv-conver]
root       243  0.0  0.0      0     0 ?        I<   Jan27   0:00 [kworker/0:1H]
root       260  0.0  0.0      0     0 ?        I<   Jan27   0:00 [kworker/2:1H]
root       266  0.0  0.0      0     0 ?        I<   Jan27   0:00 [rpciod]
root       267  0.0  0.0      0     0 ?        I<   Jan27   0:00 [xprtiod]
root       287  0.0  0.4 127564 36564 ?        S<s  Jan27   0:00 /lib/systemd/systemd-journald
root       297  0.0  0.0  23920   180 ?        Ss   Jan27   0:00 /usr/sbin/blkmapd
root       308  0.0  0.0  46008  4836 ?        Ss   Jan27   0:00 /lib/systemd/systemd-udevd
root       366  0.0  0.0      0     0 ?        I<   Jan27   0:00 [ttm_swap]
systemd+   527  0.0  0.0 141932  3228 ?        Ssl  Jan27   0:00 /lib/systemd/systemd-timesyncd
root       530  0.0  0.0  47604  3504 ?        Ss   Jan27   0:00 /sbin/rpcbind -f -w
root       534  0.0  0.0  32344  3280 ?        Ss   Jan27   0:00 /usr/sbin/cron -f
daemon     536  0.0  0.0  28332  2232 ?        Ss   Jan27   0:00 /usr/sbin/atd -f
syslog     545  0.0  0.0 263036  4632 ?        Ssl  Jan27   0:00 /usr/sbin/rsyslogd -n
root       556  0.0  0.0 110484  3420 ?        Ssl  Jan27   0:00 /usr/sbin/irqbalance --foreground
root       561  0.0  0.0 288680  7136 ?        Ssl  Jan27   0:00 /usr/lib/accountsservice/accounts-daemon
message+   569  0.0  0.0  50104  4188 ?        Ss   Jan27   0:00 /usr/bin/dbus-daemon --system --address=systemd: --nofork --nopidfile
root       586  0.0  0.0  45232  2956 ?        Ss   Jan27   0:00 /sbin/wpa_supplicant -u -s -O /run/wpa_supplicant
root       589  0.0  0.0  70600  6060 ?        Ss   Jan27   0:00 /lib/systemd/systemd-logind
root       592  0.0  0.2 171560 17528 ?        Ssl  Jan27   0:00 /usr/bin/python3 /usr/bin/networkd-dispatcher --run-startup-triggers
root       607  0.0  0.0 288884  6344 ?        Ssl  Jan27   0:00 /usr/lib/policykit-1/polkitd --no-debug
systemd+   634  0.0  0.0  70636  5320 ?        Ss   Jan27   0:00 /lib/systemd/systemd-resolved
root       663  0.0  0.0  25992  3592 ?        Ss   Jan27   0:00 /sbin/dhclient -1 -4 -v -pf /run/dhclient.eth1.pid -lf /var/lib/dhcp/d
root       777  0.0  0.1 270744 12652 ?        Ss   Jan27   0:00 /usr/sbin/nmbd --foreground --no-process-group
root       781  0.0  0.2 188264 20376 ?        Ssl  Jan27   0:00 /usr/bin/python3 /usr/share/unattended-upgrades/unattended-upgrade-shu
root       783  0.0  0.0  72300  5636 ?        Ss   Jan27   0:00 /usr/sbin/sshd -D
root       787  0.0  0.0  17204  1912 tty1     Ss+  Jan27   0:00 /sbin/agetty -o -p -- \u --noclear tty1 linux
root       819  0.0  0.2 356924 20736 ?        Ss   Jan27   0:00 /usr/sbin/smbd --foreground --no-process-group
root       821  0.0  0.0 346060  6052 ?        S    Jan27   0:00 /usr/sbin/smbd --foreground --no-process-group
root       822  0.0  0.0 346052  4660 ?        S    Jan27   0:00 /usr/sbin/smbd --foreground --no-process-group
root       823  0.0  0.0 356924  7144 ?        S    Jan27   0:00 /usr/sbin/smbd --foreground --no-process-group
root      1920  0.0  0.0      0     0 ?        I    Jan27   0:00 [kworker/1:0]
root      2843  0.0  0.0      0     0 ?        I    01:21   0:00 [kworker/0:0]
root      2844  0.0  0.0      0     0 ?        I    01:21   0:00 [kworker/3:0]
root      3009  0.0  0.0 103864  7180 ?        Ss   02:47   0:00 sshd: tatsu [priv]
tatsu     3032  0.0  0.0  76648  7624 ?        Ss   02:47   0:00 /lib/systemd/systemd --user
tatsu     3033  0.0  0.0 257408  2624 ?        S    02:47   0:00 (sd-pam)
tatsu     3142  0.0  0.0 103864  3392 ?        S    02:47   0:00 sshd: tatsu@pts/0
tatsu     3143  0.0  0.0  51116  7092 pts/0    Ss   02:47   0:00 -zsh
root      3520  0.0  0.0      0     0 ?        I    03:11   0:00 [kworker/u8:1]
root      3555  0.0  0.0      0     0 ?        I    03:17   0:00 [kworker/2:0]
root      3592  0.0  0.0      0     0 ?        I    03:37   0:00 [kworker/u8:2]
root      3656  0.0  0.0      0     0 ?        I    03:47   0:00 [kworker/u8:0]
tatsu     3678  0.0  0.0  38392  3676 pts/0    R+   03:48   0:00 ps aux
    \end{verbatim}
\end{itembox}

\subsubsection{特定のプロセス名の取得}

以上の結果に加えて,ユーザーが独自に起動したプロセスの情報を取得できているかを下に示す.
評価としては,\verb|user|という無限ループするのみのユーザープロセスを起動し,その情報を取得する.

以下に実行結果を示す.

\begin{itembox}[l]{process-listからuserというプロセスを検索}
    \begin{verbatim}
./process-list 0xFFFFFFFF82413480 | grep user
0x00000234ba96c0   4189 S: user
    \end{verbatim}
\end{itembox}

\begin{itembox}[l]{監視対象ホストのpsコマンドからuserというプロセスを検索}
    \begin{verbatim}
ps aux | grep "user"
tatsu     3032  0.0  0.0  76648  7624 ?        Ss   02:47   0:00 /lib/systemd/systemd --user
tatsu     4189  0.0  0.0   4508   808 pts/1    S+   03:51   0:00 ./user
tatsu     4207  0.0  0.0  15452  1004 pts/0    S+   03:52   0:00 grep --color=auto --exclude-dir=.bzr --exclude-dir=CVS --exclude-dir=.git --exclude-dir=.hg --exclude-dir=.svn user
    \end{verbatim}
\end{itembox}

プロセスIDとして4189を持つプロセスがどちらにも存在しているため,正しくプロセスの情報を取得できていることを示せた.

\subsection{カーネルパニック発生時における評価}

カーネルパニック発生時は既存の手法,プロセスとして起動する方法はだめだが,本研究における実装では問題なく動作することを示す.

\section{評価のまとめ}

本研究の実装に対する評価として,\ref{chap:implementation}章で述べた構成の元,ネットワーク越しに存在している物理的なマシンのプロセス情報の一覧を正しく取得することで,
RDMAを用いたメモリ探索を行うことで,オペレーティングシステムの復元をすることが可能であることを示した.

また,オペレーティングシステムのコンテキストの復元を行う上で,監視対象ホストのカーネルコンフィグに依存する情報を正しく復元できていることを示した.

% カーネルパニックのくだりはどうする.
