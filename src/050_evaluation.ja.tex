\chapter{評価}
\label{chap:evaluation}

\section{評価手法}

カーネルのバージョンのみわかる状態から,正しくps auxと同じような出力を得られるかどうか,
また,\verb|task_struct|構造体の構成に関わるようなカーネルコンフィグを変更し,変更後でも差異を吸収して上記の結果を得られるかどうか

プロセスとして監視を行う手法と,本研究の実装を,通常稼働中とカーネルパニック発生時における実行の可否について述べる.

\section{評価}

\subsection{カーネルコンフィグ変更時の動作確認}

コンフィグいじってビルドし直したものに対しても問題なく解析を行えるかどうか.

\subsection{通常稼働中における評価}

通常時において,正しい出力を得られていることを示す.

\subsection{カーネルパニック発生時における評価}

カーネルパニック発生時は既存の手法,プロセスとして起動する方法はだめだが,本研究における実装では問題なく動作することを示す.

\subsection{時間}

初期段階で時間はかかるが,それは問題ではない.大事なのは,どんなカーネルコンフィグを持つLinuxでも,解析が可能になるという点.

\section{評価}

未評価
